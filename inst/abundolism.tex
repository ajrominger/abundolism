% Options for packages loaded elsewhere
% Options for packages loaded elsewhere
\PassOptionsToPackage{unicode}{hyperref}
\PassOptionsToPackage{hyphens}{url}
\PassOptionsToPackage{dvipsnames,svgnames,x11names}{xcolor}
%
\documentclass[
  letterpaper,
  DIV=11,
  numbers=noendperiod]{scrartcl}
\usepackage{xcolor}
\usepackage{amsmath,amssymb}
\setcounter{secnumdepth}{-\maxdimen} % remove section numbering
\usepackage{iftex}
\ifPDFTeX
  \usepackage[T1]{fontenc}
  \usepackage[utf8]{inputenc}
  \usepackage{textcomp} % provide euro and other symbols
\else % if luatex or xetex
  \usepackage{unicode-math} % this also loads fontspec
  \defaultfontfeatures{Scale=MatchLowercase}
  \defaultfontfeatures[\rmfamily]{Ligatures=TeX,Scale=1}
\fi
\usepackage{lmodern}
\ifPDFTeX\else
  % xetex/luatex font selection
\fi
% Use upquote if available, for straight quotes in verbatim environments
\IfFileExists{upquote.sty}{\usepackage{upquote}}{}
\IfFileExists{microtype.sty}{% use microtype if available
  \usepackage[]{microtype}
  \UseMicrotypeSet[protrusion]{basicmath} % disable protrusion for tt fonts
}{}
\makeatletter
\@ifundefined{KOMAClassName}{% if non-KOMA class
  \IfFileExists{parskip.sty}{%
    \usepackage{parskip}
  }{% else
    \setlength{\parindent}{0pt}
    \setlength{\parskip}{6pt plus 2pt minus 1pt}}
}{% if KOMA class
  \KOMAoptions{parskip=half}}
\makeatother
% Make \paragraph and \subparagraph free-standing
\makeatletter
\ifx\paragraph\undefined\else
  \let\oldparagraph\paragraph
  \renewcommand{\paragraph}{
    \@ifstar
      \xxxParagraphStar
      \xxxParagraphNoStar
  }
  \newcommand{\xxxParagraphStar}[1]{\oldparagraph*{#1}\mbox{}}
  \newcommand{\xxxParagraphNoStar}[1]{\oldparagraph{#1}\mbox{}}
\fi
\ifx\subparagraph\undefined\else
  \let\oldsubparagraph\subparagraph
  \renewcommand{\subparagraph}{
    \@ifstar
      \xxxSubParagraphStar
      \xxxSubParagraphNoStar
  }
  \newcommand{\xxxSubParagraphStar}[1]{\oldsubparagraph*{#1}\mbox{}}
  \newcommand{\xxxSubParagraphNoStar}[1]{\oldsubparagraph{#1}\mbox{}}
\fi
\makeatother


\usepackage{longtable,booktabs,array}
\usepackage{calc} % for calculating minipage widths
% Correct order of tables after \paragraph or \subparagraph
\usepackage{etoolbox}
\makeatletter
\patchcmd\longtable{\par}{\if@noskipsec\mbox{}\fi\par}{}{}
\makeatother
% Allow footnotes in longtable head/foot
\IfFileExists{footnotehyper.sty}{\usepackage{footnotehyper}}{\usepackage{footnote}}
\makesavenoteenv{longtable}
\usepackage{graphicx}
\makeatletter
\newsavebox\pandoc@box
\newcommand*\pandocbounded[1]{% scales image to fit in text height/width
  \sbox\pandoc@box{#1}%
  \Gscale@div\@tempa{\textheight}{\dimexpr\ht\pandoc@box+\dp\pandoc@box\relax}%
  \Gscale@div\@tempb{\linewidth}{\wd\pandoc@box}%
  \ifdim\@tempb\p@<\@tempa\p@\let\@tempa\@tempb\fi% select the smaller of both
  \ifdim\@tempa\p@<\p@\scalebox{\@tempa}{\usebox\pandoc@box}%
  \else\usebox{\pandoc@box}%
  \fi%
}
% Set default figure placement to htbp
\def\fps@figure{htbp}
\makeatother


% definitions for citeproc citations
\NewDocumentCommand\citeproctext{}{}
\NewDocumentCommand\citeproc{mm}{%
  \begingroup\def\citeproctext{#2}\cite{#1}\endgroup}
\makeatletter
 % allow citations to break across lines
 \let\@cite@ofmt\@firstofone
 % avoid brackets around text for \cite:
 \def\@biblabel#1{}
 \def\@cite#1#2{{#1\if@tempswa , #2\fi}}
\makeatother
\newlength{\cslhangindent}
\setlength{\cslhangindent}{1.5em}
\newlength{\csllabelwidth}
\setlength{\csllabelwidth}{3em}
\newenvironment{CSLReferences}[2] % #1 hanging-indent, #2 entry-spacing
 {\begin{list}{}{%
  \setlength{\itemindent}{0pt}
  \setlength{\leftmargin}{0pt}
  \setlength{\parsep}{0pt}
  % turn on hanging indent if param 1 is 1
  \ifodd #1
   \setlength{\leftmargin}{\cslhangindent}
   \setlength{\itemindent}{-1\cslhangindent}
  \fi
  % set entry spacing
  \setlength{\itemsep}{#2\baselineskip}}}
 {\end{list}}
\usepackage{calc}
\newcommand{\CSLBlock}[1]{\hfill\break\parbox[t]{\linewidth}{\strut\ignorespaces#1\strut}}
\newcommand{\CSLLeftMargin}[1]{\parbox[t]{\csllabelwidth}{\strut#1\strut}}
\newcommand{\CSLRightInline}[1]{\parbox[t]{\linewidth - \csllabelwidth}{\strut#1\strut}}
\newcommand{\CSLIndent}[1]{\hspace{\cslhangindent}#1}



\setlength{\emergencystretch}{3em} % prevent overfull lines

\providecommand{\tightlist}{%
  \setlength{\itemsep}{0pt}\setlength{\parskip}{0pt}}



 


\usepackage{booktabs}
\usepackage{longtable}
\usepackage{array}
\usepackage{multirow}
\usepackage{wrapfig}
\usepackage{float}
\usepackage{colortbl}
\usepackage{pdflscape}
\usepackage{tabu}
\usepackage{threeparttable}
\usepackage{threeparttablex}
\usepackage[normalem]{ulem}
\usepackage{makecell}
\usepackage{xcolor}
\usepackage{xr}
\externaldocument{ab_supp}
\newcommand{\sfig}[1]{Supplementary Figure~\ref{#1}}
\newcommand{\ssec}[1]{Supplementary Section~\ref{#1}}
\KOMAoption{captions}{tableheading}
\makeatletter
\@ifpackageloaded{caption}{}{\usepackage{caption}}
\AtBeginDocument{%
\ifdefined\contentsname
  \renewcommand*\contentsname{Table of contents}
\else
  \newcommand\contentsname{Table of contents}
\fi
\ifdefined\listfigurename
  \renewcommand*\listfigurename{List of Figures}
\else
  \newcommand\listfigurename{List of Figures}
\fi
\ifdefined\listtablename
  \renewcommand*\listtablename{List of Tables}
\else
  \newcommand\listtablename{List of Tables}
\fi
\ifdefined\figurename
  \renewcommand*\figurename{Figure}
\else
  \newcommand\figurename{Figure}
\fi
\ifdefined\tablename
  \renewcommand*\tablename{Table}
\else
  \newcommand\tablename{Table}
\fi
}
\@ifpackageloaded{float}{}{\usepackage{float}}
\floatstyle{ruled}
\@ifundefined{c@chapter}{\newfloat{codelisting}{h}{lop}}{\newfloat{codelisting}{h}{lop}[chapter]}
\floatname{codelisting}{Listing}
\newcommand*\listoflistings{\listof{codelisting}{List of Listings}}
\makeatother
\makeatletter
\makeatother
\makeatletter
\@ifpackageloaded{caption}{}{\usepackage{caption}}
\@ifpackageloaded{subcaption}{}{\usepackage{subcaption}}
\makeatother
\usepackage{bookmark}
\IfFileExists{xurl.sty}{\usepackage{xurl}}{} % add URL line breaks if available
\urlstyle{same}
\hypersetup{
  pdftitle={Intermediate abundance promotes speciation when dispersal is limited},
  pdfauthor={Andrew J. Rominger; Luke J. Harmon; Isaac Overcast; Christine E. Parent; James L. Rosindell; Catherine E. Wagner},
  colorlinks=true,
  linkcolor={blue},
  filecolor={Maroon},
  citecolor={Blue},
  urlcolor={Blue},
  pdfcreator={LaTeX via pandoc}}


\title{Intermediate abundance promotes speciation when dispersal is
limited}
\author{Andrew J. Rominger \and Luke J. Harmon \and Isaac
Overcast \and Christine E. Parent \and James L. Rosindell \and Catherine
E. Wagner}
\date{}
\begin{document}
\maketitle


\subsection{Abstract}\label{abstract}

Stub

\subsection{Introduction}\label{introduction}

A fundamental question in evolutionary ecology concerns why some clades
are diverse while others are not. Are there properties of species and
lineages that promote speciation? One line of inquiry with a long and
convoluted history is the role of rarity and commonness is driving
diversification. Going back over 150 years, Darwin (1859) argued that
widespread, abundant species should be superior competitors and thus
more likely to give rise to new species.

This line of thinking carried forward to macroecologists who attempted
to model, with some empirical support, the presumed positive association
between commonness, fitness, and speciation (Maurer 1999).
Paleontologists searched for relationships between commonness and
diversification, alternately finding evidence for a positive
relationship (Krug et al. 2008) but also a negative relationship
{[}stanley1986; Jablonski and Roy (2003){]}. Empirical studies of modern
taxa have largely found that commonness is negatively or inconclusively
related to measures of diversification (Jablonski and Roy 2003,
Makarieva and Gorshkov 2004, Greenberg and Mooers 2017, Smyčka et al.
2023, Afonso Silva et al. 2025, but see Hay et al. 2022). While some of
these studies focus specifically on average abundance, others on range
size, and some on both, the two macroecological properties are most
often strongly correlated (Brown 1995, Gaston 2003) and for now we will
treat both as existing on a continuum between rarity and commonness.

More recent theoretical and modeling work has largely assumed a positive
relationship between commonness and diversification. Hubbell (2001)
perhaps started this trend with the inclusion of ``point mutation''
speciation in the Unified Neutral Theory of Biodiversity (UNTB). Because
point mutation speciation assumes a constant probability of specaition
for every individual, lineages with more individuals will experience
more speciation events in the UNTB {[}Hubbell (2001);
etienne2007modes{]}. The speciation mechanism first proposed by Hubbell
(2001) was never an accurate model of real diversification, but even
later attempts to increase the realism of speciation in the UNTB retain
the emergent property that more abundant lineages will undergo more
specaition: protracted speciation (Rosindell et al. 2010) again assumes
that incipient speciation is constant across individuals and thus full
speciation will scale positively with lineage abundance; fission
speciation (Etienne and Haegeman 2011) also assumes the probability of
fission (the event that leads to speciation) increases with the number
of individuals in a species. Largely independently from UNTB and its
descendants, phylogenetic models of geographic change and
diversification also explicitly assume that more widespread species have
greater opportunity for speciation (Goldberg et al. 2011).

However, another process underlies speciation itself, the study of
speciation in relation to commonness, and the relationship between
abundance and range size: dispersal. Dispersal is a key mechanism by
which populations can become isolated or connected, potentially leading
to specaition or admixture (Yamaguchi 2022); it is a key process in
phylogenetic models of geographic change and diversification (Matzke
2014); it is necessary to maintain biodiversity in the UNTB (Hubbell
2001); and it is the mechanism connecting abundance to range size in
both neutral (Hubbell 2001) and non-neutral (Brown 1995) macroecological
models.

Due to the potential for dispersal to create population isolates but
also swamp out regional differences that could have led to specaition,
it remains an intriguing question what the connection is between
dispersal ability as a biological trait and speciation as an
evolutionary outcome (Yamaguchi 2022). Some empirical studies have found
that measures of diversification have a hump-shaped or negative
relationship with morphological proxies for dispersal ability (Price and
Wagner 2004, Claramunt et al. 2012, Czekanski-Moir and Rundell 2019)
while others have found a positive to flat relationship between
dispersal ability and diversification (Claramunt et al. 2025).
Agent-based model simulations have supported the idea of a trade-off
between too little dispersal leading to population instability and too
much dispersal leading to admixture with intermediate dispersal
balancing the two extremes allowing for speciation (Birand et al. 2012,
Ashby et al. 2020, Ciccheto et al. 2024).

But dispersal is not just a trait connected to morphology or imposed in
a simulation, it emerges from population dynamics: each individual
carries some probability of dispersing, thus species with more
individuals will present with higher dispersal. This is the assumption,
a realistic one, in birth-death-immigration models (Kendall 1948) of
which the UNTB is one {[}Hubbell (2001); alonso2004{]}. Therefore,
commonness and dispersal are interconnected and interdependent in their
effects on speciation if they indeed have any consistent effects. It
also remains an open debate whether their effects result from the
determinism of adaptive evolution as Darwin (1859) believed or instead
emerge from chance.

Here we develop a birth-death-immigration model with protracted
specaition embedded in a landscape of multiple local populations
connected with limited dispersal to investigate the role of abundance
and chance in modulating the probability of speciation. Contrary to
Darwin (1859) and UNTB (Hubbell 2001, Etienne et al. 2007, Rosindell et
al. 2010, Etienne and Haegeman 2011), we find that intermediate
abundances lead to the greatest probability for dispersal. Critically,
this result does not depend in any way on whether populations are more
or less adaptively fit, it only depends on a balance between large
enough population sizes to accrue sufficient probability of incipient
speciation but small enough population sizes to not loose regional
differentiation due to increased dispersal. We also analyze real data on
the species richness and abundance of endemic arthropods in the pae
ʻāina Hawaiʻi finding empirical evidence for intermediate abundance
promoting speciation.

\subsection{Methods}\label{methods}

\subsubsection{Simulating a birth-death-immigration process with
speciation}\label{simulating-a-birth-death-immigration-process-with-speciation}

We simulate a birth-death-immigration process (BDI) with speciation
(BDIS) in a modified metapopulation (Hanski 1998) setting. In this
setting there are a number of local populations connected by dispersal
as well as an global source pool connected to each local population by
dispersal. The global source pool is considered extremely large relative
to local populations such that extinction of the global source pool is
unlikely on the time scale of local dynamics (we approximate this
assumption by never allowing the source pool to go extinct). Local
populations grow and shrink according to local births and deaths,
dispersal between local populations, and dispersal from the source pool.
New species arise via a modified protracted speciation process
(Rosindell et al. 2010) similar to the model of Tao et al. (2021). Our
simulation model is consistent with other models derived from the
Unified Neutral Theory of Biodiversity {[}UNTB; Hubbell (2001){]}, but
unlike those models' multi-species perspectives, we only concern
ourselves with a single species and whether that one species undergoes a
speciation event. In that way we do not in fact make the assumption of
per capita neutrality across species, meaning our results could be
generalization to multi-species eco-evolutionary models based on niche
theories (\textbf{niche-stuff?}).

Our simulation proceeds according to these rules:

\begin{enumerate}
\def\labelenumi{\arabic{enumi}.}
\tightlist
\item
  Birth, death, immigration, and speciation all happen independently and
  are determined by their own respective rates
\item
  Speciation has 2 steps:

  \begin{enumerate}
  \def\labelenumii{\roman{enumii}.}
  \tightlist
  \item
    incipient speciation happens by turning one local population into an
    incipient new species
  \item
    if the incipient species lasts long enough it becomes a new species;
    ``long engough'' is determined by a waiting time parameter
    \(\tau\)---the larger \(\tau\), the longer an incipient species must
    wait before becoming a genuinely new species
  \end{enumerate}
\item
  Immigration between local communities and from the global source pool
  slows the progress toward speciation; if an incipient species has to
  wait \(\tau\) time (in the absence of immigration) until it is a full
  species, each immigration event adds an increment to \(\tau\) of
  \(\xi / n_i\) where \(\xi\) is a parameter we can set and \(n_i\) is
  the population size of the incipient species
\item
  Once full speciation occurs the simulation is stopped; if the
  simulation reaches the maximum designated number of iterations without
  full speciation, then the simulation stops anyway
\end{enumerate}

\begin{figure}

\centering{

\pandocbounded{\includegraphics[keepaspectratio]{abundolism_files/figure-pdf/fig-concept-1.pdf}}

}

\caption{\label{fig-concept}Conceptual overview of our simulation
model.}

\end{figure}%

Full details of the simulation are provided in \ssec{sec-sim}.
Underlying C++ and R source code is available in the R package
\emph{abondolism} (Rominger 2026) accompanying this paper.

\subsubsection{Simulation experiment}\label{simulation-experiment}

Table~\ref{tbl-pars} summarizes the simulation parameters in our model
and the range of values we use in our simulation experiment. In all
cases parameter values are drawn from a uniform distribution. A total of
10 simulations were run, each with a unique set of randomly sampled
parameters.

\begin{table}

\caption{\label{tbl-pars}Parameters govering the simulation and their
ranges over the simulation experiment. Mathematical parameter names are
given first followed by their name in the simulation code. Note
\texttt{np} and \texttt{nstep} were fixed throughout the course of the
simulation experiment}

\centering{

\begin{tabular}{rrrr>{\raggedright\arraybackslash}p{3in}}
\toprule
\multicolumn{2}{c}{parameter} & \multicolumn{2}{c}{range} & \multicolumn{1}{c}{description} \\
\cmidrule(l{3pt}r{3pt}){1-2} \cmidrule(l{3pt}r{3pt}){3-4} \cmidrule(l{3pt}r{3pt}){5-5}
$\lambda$ & (\texttt{la}) & 0.1 & 10.000 & local birth rate\\
$\mu$ & (\texttt{mu}) & 0.1 & 10.000 & local death rate\\
$\gamma$ & (\texttt{g}) & 0.0 & 0.100 & dispersal rate from the global source pool to one local population\\
$m_\text{prop}$ & (\texttt{m\_prop}) & 0.0 & 0.100 & proportional dispersal rate between local populations relative to global rate\\
$\nu$ & (\texttt{nu}) & 0.0 & 0.001 & incipient speciation rate\\
\addlinespace
$\tau$ & (\texttt{tau}) & 0.0 & 2.000 & wait time to full speciation in the absence of dispersal\\
$\xi$ & (\texttt{xi}) & 1.0 & 1.000 & amount each dispersal event sets back the progression toward speciation\\
$n_p$ & (\texttt{np}) & 2.0 & 2.000 & number of local populations\\
$n_\text{step}$ & (\texttt{nstep}) & 10,000.0 & 10,000.000 & number of simulation iterations\\
\bottomrule
\end{tabular}

}

\end{table}%

Parameter values were chosen as a balance between biological realism and
computational efficiency: rates are sufficiently fast to allow for
abundances to reach levels often found in survey data (e.g.~the
arthropod survey data we analyze here) within

\subsubsection{Analysis of arthropod richness and abundance from the pae
ʻāina
Hawaiʻi}\label{analysis-of-arthropod-richness-and-abundance-from-the-pae-ux2bbux101ina-hawaiux2bbi}

Gruner abundance: calculate mean across sites

Bishop checklist: \emph{generally} each native genus got here once and
radiated

\subsection{Results}\label{results}

\begin{figure}

\centering{

\pandocbounded{\includegraphics[keepaspectratio]{abundolism_files/figure-pdf/fig-sim-1.pdf}}

}

\caption{\label{fig-sim}Relationship across simulation runs between
average abundance over local populations and speciation. Curve is a
quadratic binomial generalize linear model and gray region shows 95\%
confidence envelope}

\end{figure}%

So we see there is a sweet spot of intermediate abundance where
speciation is most likely to take place!

\subsection{To-do}\label{to-do}

\begin{itemize}
\tightlist
\item
  intro
\item
  bring in HI arth data
\item
  figure out if want to model arth data with stan
\item
  if yes, connect prob(speciation) to rate (r = delta\_t p o(delta\_t))
  and connect that to neg binom regression
\end{itemize}

\subsection*{References}\label{references}
\addcontentsline{toc}{subsection}{References}

\phantomsection\label{refs}
\begin{CSLReferences}{1}{1}
\bibitem[\citeproctext]{ref-afonso2025}
Afonso Silva, A. C., Maliet, O., Aristide, L., Nogués-Bravo, D., Upham,
N., Jetz, W. and Morlon, H. 2025. Negative global-scale association
between genetic diversity and speciation rates in mammals. - Nature
communications 16: 1796.

\bibitem[\citeproctext]{ref-ashby2020}
Ashby, B., Shaw, A. K. and Kokko, H. 2020. An inordinate fondness for
species with intermediate dispersal abilities. - Oikos 129: 311--319.

\bibitem[\citeproctext]{ref-birand2012}
Birand, A., Vose, A. and Gavrilets, S. 2012. Patterns of species ranges,
speciation, and extinction. - The American Naturalist 179: 1--21.

\bibitem[\citeproctext]{ref-brown1995}
Brown, J. H. 1995. Macroecology. - University of Chicago Press.

\bibitem[\citeproctext]{ref-ciccheto2024}
Ciccheto, J. R. M., Carnaval, A. C. and Araujo, S. B. L. 2024. The
influence of fragmented landscapes on speciation. - Journal of
Evolutionary Biology 37: 1499--1509.

\bibitem[\citeproctext]{ref-claramunt2012}
Claramunt, S., Derryberry, E. P., Remsen Jr, J. and Brumfield, R. T.
2012. High dispersal ability inhibits speciation in a continental
radiation of passerine birds. - Proceedings of the Royal Society B:
Biological Sciences 279: 1567--1574.

\bibitem[\citeproctext]{ref-claramunt2025}
Claramunt, S., Sheard, C., Brown, J. W., Cortés-Ramı́rez, G., Cracraft,
J., Su, M. M., Weeks, B. C. and Tobias, J. A. 2025. A new time tree of
birds reveals the interplay between dispersal, geographic range size,
and diversification. - Current Biology 35: 3883--3895.

\bibitem[\citeproctext]{ref-czekanski2019}
Czekanski-Moir, J. E. and Rundell, R. J. 2019. The ecology of
nonecological speciation and nonadaptive radiations. - Trends in Ecology
\& Evolution 34: 400--415.

\bibitem[\citeproctext]{ref-darwin}
Darwin, C. 1859. On the origin of species by means of natural selection,
or the preservation of favoured races in the struggle for life. - John
Murray.

\bibitem[\citeproctext]{ref-etienne2011fission}
Etienne, R. S. and Haegeman, B. 2011. The neutral theory of biodiversity
with random fission speciation. - Theoretical Ecology 4: 87--109.

\bibitem[\citeproctext]{ref-etienne2007modes}
Etienne, R. S., Apol, M. E. F., Olff, H. and Weissing, F. J. 2007. Modes
of speciation and the neutral theory of biodiversity. - Oikos 116:
241--258.

\bibitem[\citeproctext]{ref-gaston2002}
Gaston, K. J. 2003. The structure and dynamics of geographic ranges. -
Oxford University Press.

\bibitem[\citeproctext]{ref-goldberg2011}
Goldberg, E. E., Lancaster, L. T. and Ree, R. H. 2011. Phylogenetic
inference of reciprocal effects between geographic range evolution and
diversification. - Systematic biology 60: 451--465.

\bibitem[\citeproctext]{ref-greenberg2017}
Greenberg, D. A. and Mooers, A. Ø. 2017. Linking speciation to
extinction: Diversification raises contemporary extinction risk in
amphibians. - Evolution Letters 1: 40--48.

\bibitem[\citeproctext]{ref-hanski1998}
Hanski, I. 1998. Metapopulation dynamics. - Nature 396: 41--49.

\bibitem[\citeproctext]{ref-hay2022}
Hay, E. M., McGee, M. D. and Chown, S. L. 2022. Geographic range size
and speciation in honeyeaters. - BMC Ecology and Evolution 22: 86.

\bibitem[\citeproctext]{ref-hubbell2001}
Hubbell, S. P. 2001. The unified neutral theory of biodiversity and
biogeography. - Princeton University Press.

\bibitem[\citeproctext]{ref-jablonski2003}
Jablonski, D. and Roy, K. 2003. Geographical range and speciation in
fossil and living molluscs. - Proceedings of the Royal Society of
London. Series B: Biological Sciences 270: 401--406.

\bibitem[\citeproctext]{ref-kendall1948}
Kendall, D. G. 1948. On some modes of population growth leading to RA
fisher's logarithmic series distribution. - Biometrika 35: 6--15.

\bibitem[\citeproctext]{ref-krug2008}
Krug, A. Z., Jablonski, D. and Valentine, J. W. 2008. Species--genus
ratios reflect a global history of diversification and range expansion
in marine bivalves. - Proceedings of the Royal Society B: Biological
Sciences 275: 1117--1123.

\bibitem[\citeproctext]{ref-makarieva2004}
Makarieva, A. M. and Gorshkov, V. G. 2004. On the dependence of
speciation rates on species abundance and characteristic population
size. - Journal of Biosciences 29: 119--128.

\bibitem[\citeproctext]{ref-matzke2014}
Matzke, N. J. 2014. Model selection in historical biogeography reveals
that founder-event speciation is a crucial process in island clades. -
Systematic biology 63: 951--970.

\bibitem[\citeproctext]{ref-maurer1999}
Maurer, B. A. 1999. Untangling ecological complexity: The macroscopic
perspective. - University of Chicago Press.

\bibitem[\citeproctext]{ref-price2004}
Price, J. P. and Wagner, W. L. 2004. Speciation in hawaiian angiosperm
lineages: Cause, consequence, and mode. - Evolution 58: 2185--2200.

\bibitem[\citeproctext]{ref-abundolism-package}
Rominger, A. J. 2026.
\href{https://doi.org/10.5281/zenodo.18410689}{{abundolism}}.

\bibitem[\citeproctext]{ref-rosindell2010protracted}
Rosindell, J., Cornell, S. J., Hubbell, S. P. and Etienne, R. S. 2010.
Protracted speciation revitalizes the neutral theory of biodiversity. -
Ecology Letters 13: 716--727.

\bibitem[\citeproctext]{ref-smyvcka2023}
Smyčka, J., Toszogyova, A. and Storch, D. 2023. The relationship between
geographic range size and rates of species diversification. - Nature
Communications 14: 5559.

\bibitem[\citeproctext]{ref-tao2021}
Tao, R., Sack, L. and Rosindell, J. 2021. Biogeographic drivers of
evolutionary radiations. - Frontiers in Ecology and Evolution 9: 644328.

\bibitem[\citeproctext]{ref-yamaguchi2022}
Yamaguchi, R. 2022. Intermediate dispersal hypothesis of species
diversity: New insights. - Ecological Research 37: 301--315.

\end{CSLReferences}




\end{document}
