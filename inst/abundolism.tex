% Options for packages loaded elsewhere
% Options for packages loaded elsewhere
\PassOptionsToPackage{unicode}{hyperref}
\PassOptionsToPackage{hyphens}{url}
\PassOptionsToPackage{dvipsnames,svgnames,x11names}{xcolor}
%
\documentclass[
  letterpaper,
  DIV=11,
  numbers=noendperiod]{scrartcl}
\usepackage{xcolor}
\usepackage{amsmath,amssymb}
\setcounter{secnumdepth}{-\maxdimen} % remove section numbering
\usepackage{iftex}
\ifPDFTeX
  \usepackage[T1]{fontenc}
  \usepackage[utf8]{inputenc}
  \usepackage{textcomp} % provide euro and other symbols
\else % if luatex or xetex
  \usepackage{unicode-math} % this also loads fontspec
  \defaultfontfeatures{Scale=MatchLowercase}
  \defaultfontfeatures[\rmfamily]{Ligatures=TeX,Scale=1}
\fi
\usepackage{lmodern}
\ifPDFTeX\else
  % xetex/luatex font selection
\fi
% Use upquote if available, for straight quotes in verbatim environments
\IfFileExists{upquote.sty}{\usepackage{upquote}}{}
\IfFileExists{microtype.sty}{% use microtype if available
  \usepackage[]{microtype}
  \UseMicrotypeSet[protrusion]{basicmath} % disable protrusion for tt fonts
}{}
\makeatletter
\@ifundefined{KOMAClassName}{% if non-KOMA class
  \IfFileExists{parskip.sty}{%
    \usepackage{parskip}
  }{% else
    \setlength{\parindent}{0pt}
    \setlength{\parskip}{6pt plus 2pt minus 1pt}}
}{% if KOMA class
  \KOMAoptions{parskip=half}}
\makeatother
% Make \paragraph and \subparagraph free-standing
\makeatletter
\ifx\paragraph\undefined\else
  \let\oldparagraph\paragraph
  \renewcommand{\paragraph}{
    \@ifstar
      \xxxParagraphStar
      \xxxParagraphNoStar
  }
  \newcommand{\xxxParagraphStar}[1]{\oldparagraph*{#1}\mbox{}}
  \newcommand{\xxxParagraphNoStar}[1]{\oldparagraph{#1}\mbox{}}
\fi
\ifx\subparagraph\undefined\else
  \let\oldsubparagraph\subparagraph
  \renewcommand{\subparagraph}{
    \@ifstar
      \xxxSubParagraphStar
      \xxxSubParagraphNoStar
  }
  \newcommand{\xxxSubParagraphStar}[1]{\oldsubparagraph*{#1}\mbox{}}
  \newcommand{\xxxSubParagraphNoStar}[1]{\oldsubparagraph{#1}\mbox{}}
\fi
\makeatother

\usepackage{color}
\usepackage{fancyvrb}
\newcommand{\VerbBar}{|}
\newcommand{\VERB}{\Verb[commandchars=\\\{\}]}
\DefineVerbatimEnvironment{Highlighting}{Verbatim}{commandchars=\\\{\}}
% Add ',fontsize=\small' for more characters per line
\usepackage{framed}
\definecolor{shadecolor}{RGB}{241,243,245}
\newenvironment{Shaded}{\begin{snugshade}}{\end{snugshade}}
\newcommand{\AlertTok}[1]{\textcolor[rgb]{0.68,0.00,0.00}{#1}}
\newcommand{\AnnotationTok}[1]{\textcolor[rgb]{0.37,0.37,0.37}{#1}}
\newcommand{\AttributeTok}[1]{\textcolor[rgb]{0.40,0.45,0.13}{#1}}
\newcommand{\BaseNTok}[1]{\textcolor[rgb]{0.68,0.00,0.00}{#1}}
\newcommand{\BuiltInTok}[1]{\textcolor[rgb]{0.00,0.23,0.31}{#1}}
\newcommand{\CharTok}[1]{\textcolor[rgb]{0.13,0.47,0.30}{#1}}
\newcommand{\CommentTok}[1]{\textcolor[rgb]{0.37,0.37,0.37}{#1}}
\newcommand{\CommentVarTok}[1]{\textcolor[rgb]{0.37,0.37,0.37}{\textit{#1}}}
\newcommand{\ConstantTok}[1]{\textcolor[rgb]{0.56,0.35,0.01}{#1}}
\newcommand{\ControlFlowTok}[1]{\textcolor[rgb]{0.00,0.23,0.31}{\textbf{#1}}}
\newcommand{\DataTypeTok}[1]{\textcolor[rgb]{0.68,0.00,0.00}{#1}}
\newcommand{\DecValTok}[1]{\textcolor[rgb]{0.68,0.00,0.00}{#1}}
\newcommand{\DocumentationTok}[1]{\textcolor[rgb]{0.37,0.37,0.37}{\textit{#1}}}
\newcommand{\ErrorTok}[1]{\textcolor[rgb]{0.68,0.00,0.00}{#1}}
\newcommand{\ExtensionTok}[1]{\textcolor[rgb]{0.00,0.23,0.31}{#1}}
\newcommand{\FloatTok}[1]{\textcolor[rgb]{0.68,0.00,0.00}{#1}}
\newcommand{\FunctionTok}[1]{\textcolor[rgb]{0.28,0.35,0.67}{#1}}
\newcommand{\ImportTok}[1]{\textcolor[rgb]{0.00,0.46,0.62}{#1}}
\newcommand{\InformationTok}[1]{\textcolor[rgb]{0.37,0.37,0.37}{#1}}
\newcommand{\KeywordTok}[1]{\textcolor[rgb]{0.00,0.23,0.31}{\textbf{#1}}}
\newcommand{\NormalTok}[1]{\textcolor[rgb]{0.00,0.23,0.31}{#1}}
\newcommand{\OperatorTok}[1]{\textcolor[rgb]{0.37,0.37,0.37}{#1}}
\newcommand{\OtherTok}[1]{\textcolor[rgb]{0.00,0.23,0.31}{#1}}
\newcommand{\PreprocessorTok}[1]{\textcolor[rgb]{0.68,0.00,0.00}{#1}}
\newcommand{\RegionMarkerTok}[1]{\textcolor[rgb]{0.00,0.23,0.31}{#1}}
\newcommand{\SpecialCharTok}[1]{\textcolor[rgb]{0.37,0.37,0.37}{#1}}
\newcommand{\SpecialStringTok}[1]{\textcolor[rgb]{0.13,0.47,0.30}{#1}}
\newcommand{\StringTok}[1]{\textcolor[rgb]{0.13,0.47,0.30}{#1}}
\newcommand{\VariableTok}[1]{\textcolor[rgb]{0.07,0.07,0.07}{#1}}
\newcommand{\VerbatimStringTok}[1]{\textcolor[rgb]{0.13,0.47,0.30}{#1}}
\newcommand{\WarningTok}[1]{\textcolor[rgb]{0.37,0.37,0.37}{\textit{#1}}}

\usepackage{longtable,booktabs,array}
\usepackage{calc} % for calculating minipage widths
% Correct order of tables after \paragraph or \subparagraph
\usepackage{etoolbox}
\makeatletter
\patchcmd\longtable{\par}{\if@noskipsec\mbox{}\fi\par}{}{}
\makeatother
% Allow footnotes in longtable head/foot
\IfFileExists{footnotehyper.sty}{\usepackage{footnotehyper}}{\usepackage{footnote}}
\makesavenoteenv{longtable}
\usepackage{graphicx}
\makeatletter
\newsavebox\pandoc@box
\newcommand*\pandocbounded[1]{% scales image to fit in text height/width
  \sbox\pandoc@box{#1}%
  \Gscale@div\@tempa{\textheight}{\dimexpr\ht\pandoc@box+\dp\pandoc@box\relax}%
  \Gscale@div\@tempb{\linewidth}{\wd\pandoc@box}%
  \ifdim\@tempb\p@<\@tempa\p@\let\@tempa\@tempb\fi% select the smaller of both
  \ifdim\@tempa\p@<\p@\scalebox{\@tempa}{\usebox\pandoc@box}%
  \else\usebox{\pandoc@box}%
  \fi%
}
% Set default figure placement to htbp
\def\fps@figure{htbp}
\makeatother


% definitions for citeproc citations
\NewDocumentCommand\citeproctext{}{}
\NewDocumentCommand\citeproc{mm}{%
  \begingroup\def\citeproctext{#2}\cite{#1}\endgroup}
\makeatletter
 % allow citations to break across lines
 \let\@cite@ofmt\@firstofone
 % avoid brackets around text for \cite:
 \def\@biblabel#1{}
 \def\@cite#1#2{{#1\if@tempswa , #2\fi}}
\makeatother
\newlength{\cslhangindent}
\setlength{\cslhangindent}{1.5em}
\newlength{\csllabelwidth}
\setlength{\csllabelwidth}{3em}
\newenvironment{CSLReferences}[2] % #1 hanging-indent, #2 entry-spacing
 {\begin{list}{}{%
  \setlength{\itemindent}{0pt}
  \setlength{\leftmargin}{0pt}
  \setlength{\parsep}{0pt}
  % turn on hanging indent if param 1 is 1
  \ifodd #1
   \setlength{\leftmargin}{\cslhangindent}
   \setlength{\itemindent}{-1\cslhangindent}
  \fi
  % set entry spacing
  \setlength{\itemsep}{#2\baselineskip}}}
 {\end{list}}
\usepackage{calc}
\newcommand{\CSLBlock}[1]{\hfill\break\parbox[t]{\linewidth}{\strut\ignorespaces#1\strut}}
\newcommand{\CSLLeftMargin}[1]{\parbox[t]{\csllabelwidth}{\strut#1\strut}}
\newcommand{\CSLRightInline}[1]{\parbox[t]{\linewidth - \csllabelwidth}{\strut#1\strut}}
\newcommand{\CSLIndent}[1]{\hspace{\cslhangindent}#1}



\setlength{\emergencystretch}{3em} % prevent overfull lines

\providecommand{\tightlist}{%
  \setlength{\itemsep}{0pt}\setlength{\parskip}{0pt}}



 


\usepackage{xr}
\externaldocument{ab_supp}
\newcommand{\sfig}[1]{Supplementary Figure~\ref{#1}}
\newcommand{\ssec}[1]{Supplementary Section~\ref{#1}}
\KOMAoption{captions}{tableheading}
\makeatletter
\@ifpackageloaded{caption}{}{\usepackage{caption}}
\AtBeginDocument{%
\ifdefined\contentsname
  \renewcommand*\contentsname{Table of contents}
\else
  \newcommand\contentsname{Table of contents}
\fi
\ifdefined\listfigurename
  \renewcommand*\listfigurename{List of Figures}
\else
  \newcommand\listfigurename{List of Figures}
\fi
\ifdefined\listtablename
  \renewcommand*\listtablename{List of Tables}
\else
  \newcommand\listtablename{List of Tables}
\fi
\ifdefined\figurename
  \renewcommand*\figurename{Figure}
\else
  \newcommand\figurename{Figure}
\fi
\ifdefined\tablename
  \renewcommand*\tablename{Table}
\else
  \newcommand\tablename{Table}
\fi
}
\@ifpackageloaded{float}{}{\usepackage{float}}
\floatstyle{ruled}
\@ifundefined{c@chapter}{\newfloat{codelisting}{h}{lop}}{\newfloat{codelisting}{h}{lop}[chapter]}
\floatname{codelisting}{Listing}
\newcommand*\listoflistings{\listof{codelisting}{List of Listings}}
\makeatother
\makeatletter
\makeatother
\makeatletter
\@ifpackageloaded{caption}{}{\usepackage{caption}}
\@ifpackageloaded{subcaption}{}{\usepackage{subcaption}}
\makeatother
\usepackage{bookmark}
\IfFileExists{xurl.sty}{\usepackage{xurl}}{} % add URL line breaks if available
\urlstyle{same}
\hypersetup{
  pdftitle={Intermediate abundance promotes speciation},
  colorlinks=true,
  linkcolor={blue},
  filecolor={Maroon},
  citecolor={Blue},
  urlcolor={Blue},
  pdfcreator={LaTeX via pandoc}}


\title{Intermediate abundance promotes speciation}
\author{}
\date{}
\begin{document}
\maketitle


\subsection{Abstract}\label{abstract}

Foo foo foo

\subsection{Introduction}\label{introduction}

\begin{Shaded}
\begin{Highlighting}[]
\FunctionTok{print}\NormalTok{(}\FunctionTok{Sys.time}\NormalTok{())}
\end{Highlighting}
\end{Shaded}

\begin{verbatim}
[1] "2026-01-28 16:19:56 HST"
\end{verbatim}

\begin{Shaded}
\begin{Highlighting}[]
\NormalTok{quarto}\SpecialCharTok{::}\FunctionTok{quarto\_render}\NormalTok{(}\StringTok{"ab\_supp.qmd"}\NormalTok{, }\AttributeTok{quiet =} \ConstantTok{TRUE}\NormalTok{)}
\NormalTok{tinytex}\SpecialCharTok{::}\FunctionTok{xelatex}\NormalTok{(}\StringTok{"ab\_supp.tex"}\NormalTok{, }\AttributeTok{clean =} \ConstantTok{FALSE}\NormalTok{)}
\end{Highlighting}
\end{Shaded}

\begin{verbatim}
[1] "ab_supp.pdf"
\end{verbatim}

WTF \sfig{fig-one}

OMG \ssec{sec-intro}

\begin{itemize}
\tightlist
\item
  Jablonski and Roy (2003): ``Late Cretaceous gastropod genera exhibit a
  strong negative relation between the geographical ranges of
  constituent species and speciation rate per species per million
  years\ldots These results support the view that the factors promoting
  broad geographical ranges also tend to damp speciation rates''
  analysis of modern only might not reveal this pattern---why
  Hawawi\textquotesingle i could be a good place to look
\item
  Krug et al. (2008): uses spp:gen ratio. widespread bival genera have
  more species. ``Species within these cosmopolitan genera may have
  life-history traits or other attributes that allow for rapid
  adaptation to new environments, promoting both speciation and range
  expansion of a lineage''
\item
  Birand et al. (2012): ABM, but about range and pixles are agents.
  found increased dispersal decreases speciation
\item
  Hay et al. (2022): Provides good background on range size +/- corr
  with speciation. ``Honeyeater speciation rate differs considerably
  between islands and the continental setting across the clade's
  distribution, with range size contributing positively in the
  continental setting, while dispersal ability influences speciation
  regardless of setting. These outcomes support Darwin's original
  proposal for a positive relationship between range size and speciation
  likelihood, while extending the evidence for the contribution of
  dispersal ability to speciation.''
\item
  Goldberg et al. (2011): GeoSSE model assumes speciation increases with
  range size, extinction decreases with it
\item
  Smyčka et al. (2023): hard to say if range size is + or - corr with
  speciation (could be impacted by change in range size with speciation)
\item
  Kisel and Barraclough (2010): gene flow important for speciation;
  islands good for studying speciation
\item
  Gaston (2003): classic reference on range size + corr with speciation
\item
  Pigot et al. (2010): model of range size and diversification: finds
  larger ranges produce more speciation, but with specaition, ranges
  shrink, causing specaition slow down
\item
  Ashby et al. (2020): ABM showing itermediate dispersal produces most
  speciation
\item
  Price and Wagner (2004): speciation in Hawai`i plants; intermediate
  disperal hypothesis; should check for possible data usability on
  lineage designation
\item
  Claramunt et al. (2012): ``Using a surrogate for flight performance
  and a species-level DNA-based phylogeny of a large South American bird
  radiation (the Furnariidae), we found that lineages with higher
  dispersal ability experienced lower speciation rates.''
\item
  Ciccheto et al. (2024): model shows intermediate dispersal is optimum
  for diversification
\item
  Casey et al. (2021): extinction up with decreasing abundance (range
  size also, but less predictive)
\item
  Claramunt et al. (2025): ``we found mixed evidence for the effect of
  dispersal on diversification rates: dispersive lineages show either
  slightly higher speciation rates or higher extinction rates. Our
  results therefore suggest that high dispersal ability increases range
  expansion and turnover, perhaps because dispersive lineages expand
  into islands or other geographically restricted environments and have
  lower population sizes\ldots even though per capita dispersal rates
  may be high for highly dispersive species, levels of gene flow may be
  relatively low due to lower population size. This effect may explain
  why population genetic studies have not consistently found the
  expected negative relationship between dispersal ability and genetic
  differentiation71,72,73 and why rates of speciation remain relatively
  stable across a wide range of dispersal capabilities''
\item
  Makarieva and Gorshkov (2004): relationship between abundance and
  speciation open question. they go into neutral pop gen which we should
  too
\item
  Afonso Silva et al. (2025): negative relationship between gen div and
  diversification. go into how low gen div, and thus Ne, could drive
  speciation
\item
  Rosindell et al. (2010): protracted speciation; still most abundant is
  speciator
\item
  Etienne and Haegeman (2011): fission speciation; still most abundant
  is speciator
\item
  Etienne et al. (2007): makes explicit ``the speciation rate of a
  species is directly proportional to its abundance in the
  metacommunity.'' Find abundance-independent speciation is poor fit to
  data, indicates speciation matters
\item
  Stanley (1990): speciation and extinction are correlated; could be
  abundance driven, dispersal driven, specialization driven
\item
  Stanley (1986): speciation, extinction correlated in Neogene bivalvs
  he was looking at; proposed a model for humped speciation across
  abundance
\item
  Darwin (1859): widespread, more common leads to more speciation
  because of superiority
\item
  Maurer (1999): organizes darwin's argument about widespread species
  being greater speciators
\item
  Brown (1995): abundance range size correlation
\item
  Czekanski-Moir and Rundell (2019): review finding that poor dispersal
  is key in producing ``nonadatpive'' radiations
\item
  Greenberg and Mooers (2017): link between extant species diversity and
  extinction risk (arguing for correlation between etinciton and
  speciation)
\end{itemize}

\subsection{Birth-death-immigration model with
speciation}\label{birth-death-immigration-model-with-speciation}

We simulate a birth-death-immigration model (BDI) with speciation (BDIS)
in a metapopulation setting. This is the set-up

\begin{itemize}
\tightlist
\item
  There are \texttt{np} number of local populations in the
  metapopulation
\item
  There is a global source pool
\end{itemize}

Here are the biological process steps:

\begin{enumerate}
\def\labelenumi{\arabic{enumi}.}
\tightlist
\item
  Birth, death, immigration, and speciation all happen indipendently and
  are determined by respective rates (see params below)
\item
  Speciation has 2 steps:

  \begin{enumerate}
  \def\labelenumii{\roman{enumii}.}
  \tightlist
  \item
    incipient speciation happens by turning one local population into an
    incipient new species
  \item
    if the incipient species lasts long enough it becomes a new species
  \end{enumerate}
\item
  Immigration between local communities and from the global source pool
  slows the progress toward speciation (technically if an incipient
  species has to wait \(\tau\) time (in the absence of immigration)
  until it's a full species, each immigration event adds an increment to
  \(\tau\) of \(\xi / n_i\) where \(\xi\) is a parameter we can set and
  \(n_i\) is the population size of the incipient species)
\item
  Once full speciation occurs the simulation is stopped; if the
  simulation reaches the maximum designated number of iterations
  (\texttt{nstep}) without full speciation, then the simulation stops
  anyway
\end{enumerate}

Here are the parameters:

\begin{itemize}
\tightlist
\item
  \(\lambda\) (\texttt{la}): birth rate
\item
  \(\mu\) (\texttt{mu}): death rate
\item
  \(\gamma\) (\texttt{g}): immigration rate from global source pool
\item
  \(m_p\) (\texttt{m\_prop}): proportional immigration rate between
  local populations; immigration rate \(m = \gamma \times m_p\)
\item
  \(\nu\) (\texttt{nu}): incipient speciation rate
\item
  \(\tau\) (\texttt{tau}): wait time to full speciation in the absence
  of immigration
\item
  \(\xi\) (\texttt{xi}): amount each immigrant sets back the progression
  toward speciation
\item
  (\texttt{np}): number of local populations
\item
  (\texttt{nstep}): number of iterations to run simulation for
\end{itemize}

And we can now actually run this thing and see what happens

\begin{Shaded}
\begin{Highlighting}[]
\FunctionTok{library}\NormalTok{(abundolism)}

\NormalTok{nrep }\OtherTok{\textless{}{-}} \DecValTok{1000} \CommentTok{\# number of different parameter combos to look at}
\NormalTok{la }\OtherTok{\textless{}{-}} \FunctionTok{runif}\NormalTok{(nrep, }\DecValTok{1}\NormalTok{, }\DecValTok{10}\NormalTok{)}
\NormalTok{mu }\OtherTok{\textless{}{-}} \FunctionTok{runif}\NormalTok{(nrep, }\DecValTok{1}\NormalTok{, }\DecValTok{10}\NormalTok{)}
\NormalTok{g }\OtherTok{\textless{}{-}}\NormalTok{ la }\SpecialCharTok{*} \FunctionTok{runif}\NormalTok{(nrep, }\DecValTok{0}\NormalTok{, }\FloatTok{0.1}\NormalTok{)}
\NormalTok{m\_prop }\OtherTok{\textless{}{-}} \FunctionTok{runif}\NormalTok{(nrep, }\DecValTok{0}\NormalTok{, }\FloatTok{0.1}\NormalTok{)}
\NormalTok{nu }\OtherTok{\textless{}{-}} \FunctionTok{runif}\NormalTok{(nrep, }\DecValTok{0}\NormalTok{, }\FloatTok{0.001}\NormalTok{)}
\NormalTok{tau }\OtherTok{\textless{}{-}} \FunctionTok{runif}\NormalTok{(nrep, }\DecValTok{0}\NormalTok{, }\DecValTok{4}\NormalTok{) }\CommentTok{\#10 / (la + mu + runif(nrep, 0, 1))}
\NormalTok{xi }\OtherTok{\textless{}{-}} \FunctionTok{rep}\NormalTok{(}\DecValTok{1}\NormalTok{, nrep)}
\NormalTok{np }\OtherTok{\textless{}{-}} \DecValTok{4}
\NormalTok{nstep }\OtherTok{\textless{}{-}} \DecValTok{100000}


\CommentTok{\# \textasciigrave{}sim\_BDI\_spec\textasciigrave{} is the workhorse function}
\NormalTok{sim\_dat }\OtherTok{\textless{}{-}} \FunctionTok{sim\_BDI\_spec}\NormalTok{(}\AttributeTok{la =}\NormalTok{ la, }\AttributeTok{mu =}\NormalTok{ mu, }\AttributeTok{g =}\NormalTok{ g, }\AttributeTok{m\_prop =}\NormalTok{ m\_prop,}
                        \AttributeTok{nu =}\NormalTok{ nu, }\AttributeTok{tau =}\NormalTok{ tau, }\AttributeTok{xi =}\NormalTok{ xi, }\AttributeTok{np =}\NormalTok{ np,}
                        \AttributeTok{nstep =}\NormalTok{ nstep)}

\FunctionTok{ggplot}\NormalTok{(sim\_dat, }
       \FunctionTok{aes}\NormalTok{(}\AttributeTok{x =}\NormalTok{ tau, speciation)) }\SpecialCharTok{+}
    \FunctionTok{geom\_pointdensity}\NormalTok{(}\AttributeTok{method =} \StringTok{"kde2d"}\NormalTok{) }\SpecialCharTok{+}
    \FunctionTok{scale\_shape\_binned}\NormalTok{() }\SpecialCharTok{+}
    \FunctionTok{scale\_color\_viridis\_c}\NormalTok{() }\SpecialCharTok{+}
    \CommentTok{\# scale\_x\_log10() +}
    \FunctionTok{geom\_smooth}\NormalTok{(}\AttributeTok{method =} \StringTok{"glm"}\NormalTok{, }
                \AttributeTok{method.args =} \FunctionTok{list}\NormalTok{(}\AttributeTok{family =} \StringTok{"binomial"}\NormalTok{), }
                \AttributeTok{color =} \StringTok{"black"}\NormalTok{)}

\FunctionTok{ggplot}\NormalTok{(sim\_dat, }\FunctionTok{aes}\NormalTok{(time, mean\_pop\_size)) }\SpecialCharTok{+}
    \FunctionTok{geom\_point}\NormalTok{(}\AttributeTok{data =} \FunctionTok{select}\NormalTok{(sim\_dat, }\SpecialCharTok{!}\NormalTok{speciation), }
               \AttributeTok{mapping =} \FunctionTok{aes}\NormalTok{(time, mean\_pop\_size), }\AttributeTok{color =} \StringTok{"gray50"}\NormalTok{) }\SpecialCharTok{+}
    \FunctionTok{geom\_pointdensity}\NormalTok{() }\SpecialCharTok{+} 
    \FunctionTok{facet\_wrap}\NormalTok{(}\FunctionTok{vars}\NormalTok{(speciation)) }\SpecialCharTok{+}
    \FunctionTok{scale\_x\_log10}\NormalTok{() }\SpecialCharTok{+}
    \FunctionTok{scale\_y\_log10}\NormalTok{() }\SpecialCharTok{+} 
    \FunctionTok{scale\_color\_viridis\_c}\NormalTok{(}\AttributeTok{trans =} \StringTok{"log10"}\NormalTok{)}

\FunctionTok{ggplot}\NormalTok{(sim\_dat, }\FunctionTok{aes}\NormalTok{(mean\_pop\_size)) }\SpecialCharTok{+}
    \FunctionTok{geom\_histogram}\NormalTok{() }\SpecialCharTok{+}
    \FunctionTok{scale\_x\_log10}\NormalTok{()}
\end{Highlighting}
\end{Shaded}

Now we can plot the results and find out how abundance relates to
speciation in this model

\begin{Shaded}
\begin{Highlighting}[]
\FunctionTok{library}\NormalTok{(ggplot2)}
\FunctionTok{library}\NormalTok{(ggpointdensity)}

\CommentTok{\# make a quadratic logistic model}
\NormalTok{sim\_dat}\SpecialCharTok{$}\NormalTok{log\_pop }\OtherTok{\textless{}{-}} \FunctionTok{log}\NormalTok{(sim\_dat}\SpecialCharTok{$}\NormalTok{mean\_pop\_size, }\DecValTok{10}\NormalTok{)}
\NormalTok{mod }\OtherTok{\textless{}{-}} \FunctionTok{glm}\NormalTok{(speciation }\SpecialCharTok{\textasciitilde{}}\NormalTok{ log\_pop }\SpecialCharTok{+} \FunctionTok{I}\NormalTok{(log\_pop}\SpecialCharTok{\^{}}\DecValTok{2}\NormalTok{), }\AttributeTok{data =}\NormalTok{ sim\_dat, }
           \AttributeTok{family =} \StringTok{"binomial"}\NormalTok{)}

\CommentTok{\# add to data}
\NormalTok{sim\_dat}\SpecialCharTok{$}\NormalTok{mod\_pred }\OtherTok{\textless{}{-}} \FunctionTok{predict}\NormalTok{(mod, }\AttributeTok{type =} \StringTok{"response"}\NormalTok{)}

\FunctionTok{ggplot}\NormalTok{(sim\_dat, }
       \FunctionTok{aes}\NormalTok{(}\AttributeTok{x =}\NormalTok{ mean\_pop\_size, }\AttributeTok{y =}\NormalTok{ speciation)) }\SpecialCharTok{+}
    \FunctionTok{geom\_pointdensity}\NormalTok{() }\SpecialCharTok{+}
    \FunctionTok{scale\_color\_viridis\_c}\NormalTok{() }\SpecialCharTok{+}
    \FunctionTok{scale\_x\_log10}\NormalTok{() }\SpecialCharTok{+}
    \FunctionTok{scale\_y\_continuous}\NormalTok{(}\AttributeTok{breaks =} \FunctionTok{c}\NormalTok{(}\DecValTok{0}\NormalTok{, }\DecValTok{1}\NormalTok{)) }\SpecialCharTok{+} 
    \FunctionTok{xlab}\NormalTok{(}\StringTok{"Mean population size"}\NormalTok{) }\SpecialCharTok{+}
    \FunctionTok{ylab}\NormalTok{(}\StringTok{"Speciation probability"}\NormalTok{) }\SpecialCharTok{+}
    \FunctionTok{theme}\NormalTok{(}\AttributeTok{panel.grid.minor.y =} \FunctionTok{element\_blank}\NormalTok{()) }\SpecialCharTok{+}
    \FunctionTok{geom\_smooth}\NormalTok{(}\AttributeTok{method =}\NormalTok{ glm, }\AttributeTok{formula =}\NormalTok{ y }\SpecialCharTok{\textasciitilde{}}\NormalTok{ x }\SpecialCharTok{+} \FunctionTok{I}\NormalTok{(x}\SpecialCharTok{\^{}}\DecValTok{2}\NormalTok{), }
                \AttributeTok{method.args =} \FunctionTok{list}\NormalTok{(}\AttributeTok{family =} \StringTok{"binomial"}\NormalTok{), }\AttributeTok{color =} \StringTok{"black"}\NormalTok{)}



\FunctionTok{ggplot}\NormalTok{(sim\_dat, }\FunctionTok{aes}\NormalTok{(}\AttributeTok{x =}\NormalTok{ mean\_pop\_size)) }\SpecialCharTok{+}
    \FunctionTok{geom\_histogram}\NormalTok{() }\SpecialCharTok{+}
    \FunctionTok{scale\_x\_continuous}\NormalTok{(}\AttributeTok{transform =} \StringTok{"log10"}\NormalTok{)}
\end{Highlighting}
\end{Shaded}

So we see there is a sweet spot of intermediate abundance where
speciation is most likely to take place!

\subsection{To-do}\label{to-do}

\begin{itemize}
\tightlist
\item
  intro
\item
  bring in HI arth data
\item
  figure out if want to model arth data with stan
\item
  if yes, connect prob(speciation) to rate (r = delta\_t p o(delta\_t))
  and connect that to neg binom regression
\end{itemize}

\phantomsection\label{refs}
\begin{CSLReferences}{1}{1}
\bibitem[\citeproctext]{ref-afonso2025}
Afonso Silva, A. C., Maliet, O., Aristide, L., Nogués-Bravo, D., Upham,
N., Jetz, W. and Morlon, H. 2025. Negative global-scale association
between genetic diversity and speciation rates in mammals. - Nature
communications 16: 1796.

\bibitem[\citeproctext]{ref-ashby2020}
Ashby, B., Shaw, A. K. and Kokko, H. 2020. An inordinate fondness for
species with intermediate dispersal abilities. - Oikos 129: 311--319.

\bibitem[\citeproctext]{ref-birand2012}
Birand, A., Vose, A. and Gavrilets, S. 2012. Patterns of species ranges,
speciation, and extinction. - The American Naturalist 179: 1--21.

\bibitem[\citeproctext]{ref-brown1995}
Brown, J. H. 1995. Macroecology. - University of Chicago Press.

\bibitem[\citeproctext]{ref-casey2021}
Casey, M. M., Saupe, E. E. and Lieberman, B. S. 2021. The effects of
geographic range size and abundance on extinction during a time of
{``sluggish''}'evolution. - Paleobiology 47: 54--67.

\bibitem[\citeproctext]{ref-ciccheto2024}
Ciccheto, J. R. M., Carnaval, A. C. and Araujo, S. B. L. 2024. The
influence of fragmented landscapes on speciation. - Journal of
Evolutionary Biology 37: 1499--1509.

\bibitem[\citeproctext]{ref-claramunt2012}
Claramunt, S., Derryberry, E. P., Remsen Jr, J. and Brumfield, R. T.
2012. High dispersal ability inhibits speciation in a continental
radiation of passerine birds. - Proceedings of the Royal Society B:
Biological Sciences 279: 1567--1574.

\bibitem[\citeproctext]{ref-claramunt2025}
Claramunt, S., Sheard, C., Brown, J. W., Cortés-Ramı́rez, G., Cracraft,
J., Su, M. M., Weeks, B. C. and Tobias, J. A. 2025. A new time tree of
birds reveals the interplay between dispersal, geographic range size,
and diversification. - Current Biology 35: 3883--3895.

\bibitem[\citeproctext]{ref-czekanski2019}
Czekanski-Moir, J. E. and Rundell, R. J. 2019. The ecology of
nonecological speciation and nonadaptive radiations. - Trends in Ecology
\& Evolution 34: 400--415.

\bibitem[\citeproctext]{ref-darwin}
Darwin, C. 1859. On the origin of species by means of natural selection,
or the preservation of favoured races in the struggle for life. - John
Murray.

\bibitem[\citeproctext]{ref-etienne2011fission}
Etienne, R. S. and Haegeman, B. 2011. The neutral theory of biodiversity
with random fission speciation. - Theoretical Ecology 4: 87--109.

\bibitem[\citeproctext]{ref-etienne2007modes}
Etienne, R. S., Apol, M. E. F., Olff, H. and Weissing, F. J. 2007. Modes
of speciation and the neutral theory of biodiversity. - Oikos 116:
241--258.

\bibitem[\citeproctext]{ref-gaston2002}
Gaston, K. J. 2003. The structure and dynamics of geographic ranges. -
Oxford University Press.

\bibitem[\citeproctext]{ref-goldberg2011}
Goldberg, E. E., Lancaster, L. T. and Ree, R. H. 2011. Phylogenetic
inference of reciprocal effects between geographic range evolution and
diversification. - Systematic biology 60: 451--465.

\bibitem[\citeproctext]{ref-greenberg2017}
Greenberg, D. A. and Mooers, A. Ø. 2017. Linking speciation to
extinction: Diversification raises contemporary extinction risk in
amphibians. - Evolution Letters 1: 40--48.

\bibitem[\citeproctext]{ref-hay2022}
Hay, E. M., McGee, M. D. and Chown, S. L. 2022. Geographic range size
and speciation in honeyeaters. - BMC Ecology and Evolution 22: 86.

\bibitem[\citeproctext]{ref-jablonski2003}
Jablonski, D. and Roy, K. 2003. Geographical range and speciation in
fossil and living molluscs. - Proceedings of the Royal Society of
London. Series B: Biological Sciences 270: 401--406.

\bibitem[\citeproctext]{ref-kisel2010}
Kisel, Y. and Barraclough, T. G. 2010. Speciation has a spatial scale
that depends on levels of gene flow. - The American Naturalist 175:
316--334.

\bibitem[\citeproctext]{ref-krug2008}
Krug, A. Z., Jablonski, D. and Valentine, J. W. 2008. Species--genus
ratios reflect a global history of diversification and range expansion
in marine bivalves. - Proceedings of the Royal Society B: Biological
Sciences 275: 1117--1123.

\bibitem[\citeproctext]{ref-makarieva2004}
Makarieva, A. M. and Gorshkov, V. G. 2004. On the dependence of
speciation rates on species abundance and characteristic population
size. - Journal of Biosciences 29: 119--128.

\bibitem[\citeproctext]{ref-maurer1999}
Maurer, B. A. 1999. Untangling ecological complexity: The macroscopic
perspective. - University of Chicago Press.

\bibitem[\citeproctext]{ref-pigot2010}
Pigot, A. L., Phillimore, A. B., Owens, I. P. and Orme, C. D. L. 2010.
The shape and temporal dynamics of phylogenetic trees arising from
geographic speciation. - Systematic biology 59: 660--673.

\bibitem[\citeproctext]{ref-price2004}
Price, J. P. and Wagner, W. L. 2004. Speciation in hawaiian angiosperm
lineages: Cause, consequence, and mode. - Evolution 58: 2185--2200.

\bibitem[\citeproctext]{ref-rosindell2010protracted}
Rosindell, J., Cornell, S. J., Hubbell, S. P. and Etienne, R. S. 2010.
Protracted speciation revitalizes the neutral theory of biodiversity. -
Ecology Letters 13: 716--727.

\bibitem[\citeproctext]{ref-smyvcka2023}
Smyčka, J., Toszogyova, A. and Storch, D. 2023. The relationship between
geographic range size and rates of species diversification. - Nature
Communications 14: 5559.

\bibitem[\citeproctext]{ref-stanley1986}
Stanley, S. M. 1986. Population size, extinction, and speciation: The
fission effect in neogene bivalvia. - Paleobiology 12: 89--110.

\bibitem[\citeproctext]{ref-stanley1990}
Stanley, S. M. 1990. The general correlation between rate of speciation
and rate of extinction: Fortuitous causal linkages. - Causes of
evolution: a paleontological perspective: 103.

\end{CSLReferences}




\end{document}
