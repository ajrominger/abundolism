% Options for packages loaded elsewhere
% Options for packages loaded elsewhere
\PassOptionsToPackage{unicode}{hyperref}
\PassOptionsToPackage{hyphens}{url}
\PassOptionsToPackage{dvipsnames,svgnames,x11names}{xcolor}
%
\documentclass[
  letterpaper,
  DIV=11,
  numbers=noendperiod]{scrartcl}
\usepackage{xcolor}
\usepackage{amsmath,amssymb}
\setcounter{secnumdepth}{5}
\usepackage{iftex}
\ifPDFTeX
  \usepackage[T1]{fontenc}
  \usepackage[utf8]{inputenc}
  \usepackage{textcomp} % provide euro and other symbols
\else % if luatex or xetex
  \usepackage{unicode-math} % this also loads fontspec
  \defaultfontfeatures{Scale=MatchLowercase}
  \defaultfontfeatures[\rmfamily]{Ligatures=TeX,Scale=1}
\fi
\usepackage{lmodern}
\ifPDFTeX\else
  % xetex/luatex font selection
\fi
% Use upquote if available, for straight quotes in verbatim environments
\IfFileExists{upquote.sty}{\usepackage{upquote}}{}
\IfFileExists{microtype.sty}{% use microtype if available
  \usepackage[]{microtype}
  \UseMicrotypeSet[protrusion]{basicmath} % disable protrusion for tt fonts
}{}
\makeatletter
\@ifundefined{KOMAClassName}{% if non-KOMA class
  \IfFileExists{parskip.sty}{%
    \usepackage{parskip}
  }{% else
    \setlength{\parindent}{0pt}
    \setlength{\parskip}{6pt plus 2pt minus 1pt}}
}{% if KOMA class
  \KOMAoptions{parskip=half}}
\makeatother
% Make \paragraph and \subparagraph free-standing
\makeatletter
\ifx\paragraph\undefined\else
  \let\oldparagraph\paragraph
  \renewcommand{\paragraph}{
    \@ifstar
      \xxxParagraphStar
      \xxxParagraphNoStar
  }
  \newcommand{\xxxParagraphStar}[1]{\oldparagraph*{#1}\mbox{}}
  \newcommand{\xxxParagraphNoStar}[1]{\oldparagraph{#1}\mbox{}}
\fi
\ifx\subparagraph\undefined\else
  \let\oldsubparagraph\subparagraph
  \renewcommand{\subparagraph}{
    \@ifstar
      \xxxSubParagraphStar
      \xxxSubParagraphNoStar
  }
  \newcommand{\xxxSubParagraphStar}[1]{\oldsubparagraph*{#1}\mbox{}}
  \newcommand{\xxxSubParagraphNoStar}[1]{\oldsubparagraph{#1}\mbox{}}
\fi
\makeatother

\usepackage{color}
\usepackage{fancyvrb}
\newcommand{\VerbBar}{|}
\newcommand{\VERB}{\Verb[commandchars=\\\{\}]}
\DefineVerbatimEnvironment{Highlighting}{Verbatim}{commandchars=\\\{\}}
% Add ',fontsize=\small' for more characters per line
\usepackage{framed}
\definecolor{shadecolor}{RGB}{241,243,245}
\newenvironment{Shaded}{\begin{snugshade}}{\end{snugshade}}
\newcommand{\AlertTok}[1]{\textcolor[rgb]{0.68,0.00,0.00}{#1}}
\newcommand{\AnnotationTok}[1]{\textcolor[rgb]{0.37,0.37,0.37}{#1}}
\newcommand{\AttributeTok}[1]{\textcolor[rgb]{0.40,0.45,0.13}{#1}}
\newcommand{\BaseNTok}[1]{\textcolor[rgb]{0.68,0.00,0.00}{#1}}
\newcommand{\BuiltInTok}[1]{\textcolor[rgb]{0.00,0.23,0.31}{#1}}
\newcommand{\CharTok}[1]{\textcolor[rgb]{0.13,0.47,0.30}{#1}}
\newcommand{\CommentTok}[1]{\textcolor[rgb]{0.37,0.37,0.37}{#1}}
\newcommand{\CommentVarTok}[1]{\textcolor[rgb]{0.37,0.37,0.37}{\textit{#1}}}
\newcommand{\ConstantTok}[1]{\textcolor[rgb]{0.56,0.35,0.01}{#1}}
\newcommand{\ControlFlowTok}[1]{\textcolor[rgb]{0.00,0.23,0.31}{\textbf{#1}}}
\newcommand{\DataTypeTok}[1]{\textcolor[rgb]{0.68,0.00,0.00}{#1}}
\newcommand{\DecValTok}[1]{\textcolor[rgb]{0.68,0.00,0.00}{#1}}
\newcommand{\DocumentationTok}[1]{\textcolor[rgb]{0.37,0.37,0.37}{\textit{#1}}}
\newcommand{\ErrorTok}[1]{\textcolor[rgb]{0.68,0.00,0.00}{#1}}
\newcommand{\ExtensionTok}[1]{\textcolor[rgb]{0.00,0.23,0.31}{#1}}
\newcommand{\FloatTok}[1]{\textcolor[rgb]{0.68,0.00,0.00}{#1}}
\newcommand{\FunctionTok}[1]{\textcolor[rgb]{0.28,0.35,0.67}{#1}}
\newcommand{\ImportTok}[1]{\textcolor[rgb]{0.00,0.46,0.62}{#1}}
\newcommand{\InformationTok}[1]{\textcolor[rgb]{0.37,0.37,0.37}{#1}}
\newcommand{\KeywordTok}[1]{\textcolor[rgb]{0.00,0.23,0.31}{\textbf{#1}}}
\newcommand{\NormalTok}[1]{\textcolor[rgb]{0.00,0.23,0.31}{#1}}
\newcommand{\OperatorTok}[1]{\textcolor[rgb]{0.37,0.37,0.37}{#1}}
\newcommand{\OtherTok}[1]{\textcolor[rgb]{0.00,0.23,0.31}{#1}}
\newcommand{\PreprocessorTok}[1]{\textcolor[rgb]{0.68,0.00,0.00}{#1}}
\newcommand{\RegionMarkerTok}[1]{\textcolor[rgb]{0.00,0.23,0.31}{#1}}
\newcommand{\SpecialCharTok}[1]{\textcolor[rgb]{0.37,0.37,0.37}{#1}}
\newcommand{\SpecialStringTok}[1]{\textcolor[rgb]{0.13,0.47,0.30}{#1}}
\newcommand{\StringTok}[1]{\textcolor[rgb]{0.13,0.47,0.30}{#1}}
\newcommand{\VariableTok}[1]{\textcolor[rgb]{0.07,0.07,0.07}{#1}}
\newcommand{\VerbatimStringTok}[1]{\textcolor[rgb]{0.13,0.47,0.30}{#1}}
\newcommand{\WarningTok}[1]{\textcolor[rgb]{0.37,0.37,0.37}{\textit{#1}}}

\usepackage{longtable,booktabs,array}
\usepackage{calc} % for calculating minipage widths
% Correct order of tables after \paragraph or \subparagraph
\usepackage{etoolbox}
\makeatletter
\patchcmd\longtable{\par}{\if@noskipsec\mbox{}\fi\par}{}{}
\makeatother
% Allow footnotes in longtable head/foot
\IfFileExists{footnotehyper.sty}{\usepackage{footnotehyper}}{\usepackage{footnote}}
\makesavenoteenv{longtable}
\usepackage{graphicx}
\makeatletter
\newsavebox\pandoc@box
\newcommand*\pandocbounded[1]{% scales image to fit in text height/width
  \sbox\pandoc@box{#1}%
  \Gscale@div\@tempa{\textheight}{\dimexpr\ht\pandoc@box+\dp\pandoc@box\relax}%
  \Gscale@div\@tempb{\linewidth}{\wd\pandoc@box}%
  \ifdim\@tempb\p@<\@tempa\p@\let\@tempa\@tempb\fi% select the smaller of both
  \ifdim\@tempa\p@<\p@\scalebox{\@tempa}{\usebox\pandoc@box}%
  \else\usebox{\pandoc@box}%
  \fi%
}
% Set default figure placement to htbp
\def\fps@figure{htbp}
\makeatother


% definitions for citeproc citations
\NewDocumentCommand\citeproctext{}{}
\NewDocumentCommand\citeproc{mm}{%
  \begingroup\def\citeproctext{#2}\cite{#1}\endgroup}
\makeatletter
 % allow citations to break across lines
 \let\@cite@ofmt\@firstofone
 % avoid brackets around text for \cite:
 \def\@biblabel#1{}
 \def\@cite#1#2{{#1\if@tempswa , #2\fi}}
\makeatother
\newlength{\cslhangindent}
\setlength{\cslhangindent}{1.5em}
\newlength{\csllabelwidth}
\setlength{\csllabelwidth}{3em}
\newenvironment{CSLReferences}[2] % #1 hanging-indent, #2 entry-spacing
 {\begin{list}{}{%
  \setlength{\itemindent}{0pt}
  \setlength{\leftmargin}{0pt}
  \setlength{\parsep}{0pt}
  % turn on hanging indent if param 1 is 1
  \ifodd #1
   \setlength{\leftmargin}{\cslhangindent}
   \setlength{\itemindent}{-1\cslhangindent}
  \fi
  % set entry spacing
  \setlength{\itemsep}{#2\baselineskip}}}
 {\end{list}}
\usepackage{calc}
\newcommand{\CSLBlock}[1]{\hfill\break\parbox[t]{\linewidth}{\strut\ignorespaces#1\strut}}
\newcommand{\CSLLeftMargin}[1]{\parbox[t]{\csllabelwidth}{\strut#1\strut}}
\newcommand{\CSLRightInline}[1]{\parbox[t]{\linewidth - \csllabelwidth}{\strut#1\strut}}
\newcommand{\CSLIndent}[1]{\hspace{\cslhangindent}#1}



\setlength{\emergencystretch}{3em} % prevent overfull lines

\providecommand{\tightlist}{%
  \setlength{\itemsep}{0pt}\setlength{\parskip}{0pt}}



 


\renewcommand{\thefigure}{S\arabic{figure}}
\renewcommand{\thesection}{S\arabic{section}}
\KOMAoption{captions}{tableheading}
\makeatletter
\@ifpackageloaded{caption}{}{\usepackage{caption}}
\AtBeginDocument{%
\ifdefined\contentsname
  \renewcommand*\contentsname{Table of contents}
\else
  \newcommand\contentsname{Table of contents}
\fi
\ifdefined\listfigurename
  \renewcommand*\listfigurename{List of Figures}
\else
  \newcommand\listfigurename{List of Figures}
\fi
\ifdefined\listtablename
  \renewcommand*\listtablename{List of Tables}
\else
  \newcommand\listtablename{List of Tables}
\fi
\ifdefined\figurename
  \renewcommand*\figurename{Supplementary Figure}
\else
  \newcommand\figurename{Supplementary Figure}
\fi
\ifdefined\tablename
  \renewcommand*\tablename{Table}
\else
  \newcommand\tablename{Table}
\fi
}
\@ifpackageloaded{float}{}{\usepackage{float}}
\floatstyle{ruled}
\@ifundefined{c@chapter}{\newfloat{codelisting}{h}{lop}}{\newfloat{codelisting}{h}{lop}[chapter]}
\floatname{codelisting}{Listing}
\newcommand*\listoflistings{\listof{codelisting}{List of Listings}}
\makeatother
\makeatletter
\makeatother
\makeatletter
\@ifpackageloaded{caption}{}{\usepackage{caption}}
\@ifpackageloaded{subcaption}{}{\usepackage{subcaption}}
\makeatother
\usepackage{bookmark}
\IfFileExists{xurl.sty}{\usepackage{xurl}}{} % add URL line breaks if available
\urlstyle{same}
\hypersetup{
  pdftitle={Supplement to: Intermediate abundance promotes speciation when dispersal is limited},
  colorlinks=true,
  linkcolor={blue},
  filecolor={Maroon},
  citecolor={Blue},
  urlcolor={Blue},
  pdfcreator={LaTeX via pandoc}}


\title{Supplement to: \emph{Intermediate abundance promotes speciation
when dispersal is limited}}
\author{}
\date{}
\begin{document}
\maketitle


\section{Computation set-up}\label{computation-set-up}

We created a make use of a custom R package \emph{abundolism} which can
be installed from github.

\begin{Shaded}
\begin{Highlighting}[]
\NormalTok{devtools}\SpecialCharTok{::}\FunctionTok{install\_github}\NormalTok{(}\StringTok{"ajrominger/abundolism"}\NormalTok{)}
\end{Highlighting}
\end{Shaded}

The following R computing environment is used for all simulations and
analyses:

\begin{Shaded}
\begin{Highlighting}[]
\FunctionTok{library}\NormalTok{(abundolism)}
\FunctionTok{library}\NormalTok{(dplyr)}
\end{Highlighting}
\end{Shaded}

\begin{verbatim}

Attaching package: 'dplyr'
\end{verbatim}

\begin{verbatim}
The following objects are masked from 'package:stats':

    filter, lag
\end{verbatim}

\begin{verbatim}
The following objects are masked from 'package:base':

    intersect, setdiff, setequal, union
\end{verbatim}

\begin{Shaded}
\begin{Highlighting}[]
\FunctionTok{library}\NormalTok{(ggplot2)}
\FunctionTok{library}\NormalTok{(ggpointdensity)}
\FunctionTok{library}\NormalTok{(knitr)}

\FunctionTok{sessionInfo}\NormalTok{() }\SpecialCharTok{|\textgreater{}} 
    \FunctionTok{print}\NormalTok{(}\AttributeTok{locale =} \ConstantTok{FALSE}\NormalTok{)}
\end{Highlighting}
\end{Shaded}

\begin{verbatim}
R version 4.5.2 (2025-10-31)
Platform: aarch64-apple-darwin20
Running under: macOS Sequoia 15.0.1

Matrix products: default
BLAS:   /System/Library/Frameworks/Accelerate.framework/Versions/A/Frameworks/vecLib.framework/Versions/A/libBLAS.dylib 
LAPACK: /Library/Frameworks/R.framework/Versions/4.5-arm64/Resources/lib/libRlapack.dylib;  LAPACK version 3.12.1

attached base packages:
[1] stats     graphics  grDevices utils     datasets  methods   base     

other attached packages:
[1] knitr_1.50           ggpointdensity_0.2.1 ggplot2_4.0.1       
[4] dplyr_1.1.4          abundolism_0.1.0    

loaded via a namespace (and not attached):
 [1] vctrs_0.6.5        cli_3.6.5          rlang_1.1.6        xfun_0.52         
 [5] generics_0.1.4     S7_0.2.1           jsonlite_2.0.0     glue_1.8.0        
 [9] htmltools_0.5.8.1  scales_1.4.0       rmarkdown_2.29     grid_4.5.2        
[13] evaluate_1.0.4     tibble_3.3.0       MASS_7.3-65        fastmap_1.2.0     
[17] yaml_2.3.10        lifecycle_1.0.4    compiler_4.5.2     RColorBrewer_1.1-3
[21] Rcpp_1.1.0         pkgconfig_2.0.3    rstudioapi_0.17.1  farver_2.1.2      
[25] digest_0.6.39      R6_2.6.1           tidyselect_1.2.1   pillar_1.11.0     
[29] magrittr_2.0.4     withr_3.0.2        tools_4.5.2        gtable_0.3.6      
\end{verbatim}

The quarto (Allaire et al. 2025) document used to generate this
supplement can be found with the R command
\texttt{system.file("ab\_supp.qmd",\ package\ =\ "abundolism")}.

\section{Simulation details}\label{sec-sim}

Foo foo foo

\section{Simulation experiment
details}\label{simulation-experiment-details}

Simulation parameters are drawn from uniform distributions with the
following minimum and maximum limits:

\begin{Shaded}
\begin{Highlighting}[]
\NormalTok{par\_range }\OtherTok{\textless{}{-}} \FunctionTok{data.frame}\NormalTok{(}
    \AttributeTok{param =} \FunctionTok{c}\NormalTok{(}\StringTok{"la"}\NormalTok{, }\StringTok{"mu"}\NormalTok{, }\StringTok{"g"}\NormalTok{, }\StringTok{"m\_prop"}\NormalTok{, }\StringTok{"nu"}\NormalTok{, }\StringTok{"tau"}\NormalTok{, }\StringTok{"xi"}\NormalTok{), }
    \AttributeTok{min   =} \FunctionTok{c}\NormalTok{(}\FloatTok{0.1}\NormalTok{,  }\FloatTok{0.1}\NormalTok{,  }\DecValTok{0}\NormalTok{,   }\DecValTok{0}\NormalTok{,        }\DecValTok{0}\NormalTok{,     }\DecValTok{0}\NormalTok{,    }\DecValTok{1}\NormalTok{), }
    \AttributeTok{max   =} \FunctionTok{c}\NormalTok{(}\DecValTok{10}\NormalTok{,   }\DecValTok{10}\NormalTok{,   }\FloatTok{0.1}\NormalTok{, }\FloatTok{0.1}\NormalTok{,      }\FloatTok{0.001}\NormalTok{, }\DecValTok{2}\NormalTok{,    }\DecValTok{1}\NormalTok{)}
\NormalTok{)}

\CommentTok{\# fixed params}
\NormalTok{np }\OtherTok{\textless{}{-}} \DecValTok{2}
\NormalTok{nstep }\OtherTok{\textless{}{-}} \DecValTok{10000}

\CommentTok{\# number of simulation replicates}
\NormalTok{nrep }\OtherTok{\textless{}{-}} \DecValTok{10}\CommentTok{\# 10000}

\CommentTok{\# format param names for printing}
\FunctionTok{mutate}\NormalTok{(par\_range, }
       \AttributeTok{param =} \FunctionTok{sprintf}\NormalTok{(}\StringTok{"\textasciigrave{}\%s\textasciigrave{}"}\NormalTok{, param)) }\SpecialCharTok{|\textgreater{}} 
    \FunctionTok{kable}\NormalTok{(}\AttributeTok{format.args =} \FunctionTok{list}\NormalTok{(}\AttributeTok{scientific =} \ConstantTok{FALSE}\NormalTok{))}
\end{Highlighting}
\end{Shaded}

\begin{longtable}[]{@{}lrr@{}}
\toprule\noalign{}
param & min & max \\
\midrule\noalign{}
\endhead
\bottomrule\noalign{}
\endlastfoot
\texttt{la} & 0.1 & 10.000 \\
\texttt{mu} & 0.1 & 10.000 \\
\texttt{g} & 0.0 & 0.100 \\
\texttt{m\_prop} & 0.0 & 0.100 \\
\texttt{nu} & 0.0 & 0.001 \\
\texttt{tau} & 0.0 & 2.000 \\
\texttt{xi} & 1.0 & 1.000 \\
\end{longtable}

The parameters governing number of populations (\texttt{np}) and number
of time steps (\texttt{nstep}) do not vary across simulations and are
set to \texttt{np\ =} 2 and \texttt{nstep\ =} \ensuremath{10^{4}}. We
simulate a total of \texttt{nrep\ =} 10 replicates, each with a unique,
randomly sampled set of parameter values.

\subsection{Running the simulation
experiment}\label{running-the-simulation-experiment}

Parameter values for each simulation replicate are stored in a
\texttt{data.frame} with one column for each parameter.

\begin{Shaded}
\begin{Highlighting}[]
\CommentTok{\# change or remove for different results}
\CommentTok{\# set.seed(123)}

\CommentTok{\# data.frame of randomly sampled param values}
\CommentTok{\# one param per column}
\NormalTok{pars }\OtherTok{\textless{}{-}} \FunctionTok{sapply}\NormalTok{(par\_range}\SpecialCharTok{$}\NormalTok{param, }\ControlFlowTok{function}\NormalTok{(p) \{}
    \FunctionTok{runif}\NormalTok{(nrep, }
\NormalTok{          par\_range[par\_range}\SpecialCharTok{$}\NormalTok{param }\SpecialCharTok{==}\NormalTok{ p, }\StringTok{"min"}\NormalTok{], }
\NormalTok{          par\_range[par\_range}\SpecialCharTok{$}\NormalTok{param }\SpecialCharTok{==}\NormalTok{ p, }\StringTok{"max"}\NormalTok{])}
\NormalTok{\}) }\SpecialCharTok{|\textgreater{}} 
    \FunctionTok{as.data.frame}\NormalTok{()}

\CommentTok{\# pars$tau \textless{}{-} 10 / (pars$la + pars$mu + runif(nrep, 0, 1))}
\end{Highlighting}
\end{Shaded}

\begin{Shaded}
\begin{Highlighting}[]
\CommentTok{\# now run the simulation, see \textasciigrave{}?sim\_BDI\_spec\textasciigrave{} for details }
\CommentTok{\# on input and output}
\NormalTok{sim\_dat }\OtherTok{\textless{}{-}} \FunctionTok{sim\_BDI\_spec}\NormalTok{(}\AttributeTok{la =}\NormalTok{ pars}\SpecialCharTok{$}\NormalTok{la, }\AttributeTok{mu =}\NormalTok{ pars}\SpecialCharTok{$}\NormalTok{mu, }\AttributeTok{g =}\NormalTok{ pars}\SpecialCharTok{$}\NormalTok{g, }
                        \AttributeTok{m\_prop =}\NormalTok{ pars}\SpecialCharTok{$}\NormalTok{m\_prop, }\AttributeTok{nu =}\NormalTok{ pars}\SpecialCharTok{$}\NormalTok{nu, }
                        \AttributeTok{tau =}\NormalTok{ pars}\SpecialCharTok{$}\NormalTok{tau, }\AttributeTok{xi =}\NormalTok{ pars}\SpecialCharTok{$}\NormalTok{xi, }
                        \AttributeTok{np =}\NormalTok{ np, }\AttributeTok{nstep =}\NormalTok{ nstep)}
\end{Highlighting}
\end{Shaded}

Now we can plot the results and find out how abundance relates to
speciation in this model

\begin{Shaded}
\begin{Highlighting}[]
\FunctionTok{ggplot}\NormalTok{(sim\_dat, }
       \FunctionTok{aes}\NormalTok{(}\AttributeTok{x =}\NormalTok{ tau, speciation)) }\SpecialCharTok{+}
    \FunctionTok{geom\_pointdensity}\NormalTok{(}\AttributeTok{method =} \StringTok{"kde2d"}\NormalTok{) }\SpecialCharTok{+}
    \FunctionTok{scale\_shape\_binned}\NormalTok{() }\SpecialCharTok{+}
    \FunctionTok{scale\_color\_viridis\_c}\NormalTok{() }\SpecialCharTok{+}
    \CommentTok{\# scale\_x\_log10() +}
    \FunctionTok{geom\_smooth}\NormalTok{(}\AttributeTok{method =} \StringTok{"glm"}\NormalTok{, }
                \AttributeTok{method.args =} \FunctionTok{list}\NormalTok{(}\AttributeTok{family =} \StringTok{"binomial"}\NormalTok{), }
                \AttributeTok{color =} \StringTok{"black"}\NormalTok{)}

\FunctionTok{ggplot}\NormalTok{(sim\_dat, }\FunctionTok{aes}\NormalTok{(niter, time)) }\SpecialCharTok{+}
    \FunctionTok{geom\_pointdensity}\NormalTok{(}\AttributeTok{method =} \StringTok{"kde2d"}\NormalTok{) }\SpecialCharTok{+}
    \FunctionTok{scale\_color\_viridis\_c}\NormalTok{() }\SpecialCharTok{+}
    \CommentTok{\# scale\_x\_log10() +}
    \FunctionTok{scale\_y\_log10}\NormalTok{() }\SpecialCharTok{+}
    \FunctionTok{facet\_wrap}\NormalTok{(}\FunctionTok{vars}\NormalTok{(speciation))}


\FunctionTok{ggplot}\NormalTok{(sim\_dat, }\FunctionTok{aes}\NormalTok{(time, mean\_pop\_size)) }\SpecialCharTok{+}
    \FunctionTok{geom\_point}\NormalTok{(}\AttributeTok{data =} \FunctionTok{select}\NormalTok{(sim\_dat, }\SpecialCharTok{!}\NormalTok{speciation), }
               \AttributeTok{mapping =} \FunctionTok{aes}\NormalTok{(time, mean\_pop\_size), }\AttributeTok{color =} \StringTok{"gray50"}\NormalTok{) }\SpecialCharTok{+}
    \FunctionTok{geom\_pointdensity}\NormalTok{() }\SpecialCharTok{+} 
    \FunctionTok{facet\_wrap}\NormalTok{(}\FunctionTok{vars}\NormalTok{(speciation)) }\SpecialCharTok{+}
    \FunctionTok{scale\_x\_log10}\NormalTok{() }\SpecialCharTok{+}
    \FunctionTok{scale\_y\_log10}\NormalTok{() }\SpecialCharTok{+} 
    \FunctionTok{scale\_color\_viridis\_c}\NormalTok{(}\AttributeTok{trans =} \StringTok{"log10"}\NormalTok{)}

\FunctionTok{ggplot}\NormalTok{(sim\_dat, }\FunctionTok{aes}\NormalTok{(time, mean\_pop\_size)) }\SpecialCharTok{+}
    \FunctionTok{geom\_point}\NormalTok{(}\AttributeTok{data =} \FunctionTok{select}\NormalTok{(sim\_dat, }\SpecialCharTok{!}\NormalTok{speciation), }
               \AttributeTok{mapping =} \FunctionTok{aes}\NormalTok{(time, mean\_pop\_size), }\AttributeTok{color =} \StringTok{"gray50"}\NormalTok{) }\SpecialCharTok{+}
    \FunctionTok{geom\_pointdensity}\NormalTok{() }\SpecialCharTok{+} 
    \FunctionTok{facet\_wrap}\NormalTok{(}\FunctionTok{vars}\NormalTok{(speciation)) }\SpecialCharTok{+}
    \FunctionTok{scale\_x\_log10}\NormalTok{() }\SpecialCharTok{+}
    \FunctionTok{scale\_y\_log10}\NormalTok{() }\SpecialCharTok{+} 
    \FunctionTok{scale\_color\_viridis\_c}\NormalTok{(}\AttributeTok{trans =} \StringTok{"log10"}\NormalTok{)}

\FunctionTok{ggplot}\NormalTok{(sim\_dat[sim\_dat}\SpecialCharTok{$}\NormalTok{time }\SpecialCharTok{\textless{}} \DecValTok{100}\NormalTok{, ], }\FunctionTok{aes}\NormalTok{(time, mean\_pop\_size)) }\SpecialCharTok{+}
    \FunctionTok{geom\_pointdensity}\NormalTok{() }\SpecialCharTok{+} 
    \FunctionTok{facet\_wrap}\NormalTok{(}\FunctionTok{vars}\NormalTok{(speciation)) }\SpecialCharTok{+}
    \FunctionTok{scale\_x\_log10}\NormalTok{() }\SpecialCharTok{+}
    \FunctionTok{scale\_y\_log10}\NormalTok{() }\SpecialCharTok{+} 
    \FunctionTok{scale\_color\_viridis\_c}\NormalTok{(}\AttributeTok{trans =} \StringTok{"log10"}\NormalTok{)}

\FunctionTok{ggplot}\NormalTok{(sim\_dat, }\FunctionTok{aes}\NormalTok{(time)) }\SpecialCharTok{+}
    \FunctionTok{geom\_histogram}\NormalTok{() }\SpecialCharTok{+}
    \FunctionTok{facet\_wrap}\NormalTok{(}\FunctionTok{vars}\NormalTok{(speciation), }\AttributeTok{ncol =} \DecValTok{1}\NormalTok{) }\SpecialCharTok{+}
    \FunctionTok{scale\_x\_log10}\NormalTok{()}
\end{Highlighting}
\end{Shaded}

This is the main figure

\begin{Shaded}
\begin{Highlighting}[]
\NormalTok{sim\_fig }\OtherTok{\textless{}{-}} \FunctionTok{ggplot}\NormalTok{(sim\_dat,}
                  \FunctionTok{aes}\NormalTok{(}\AttributeTok{x =}\NormalTok{ mean\_pop\_size, }\AttributeTok{y =}\NormalTok{ speciation)) }\SpecialCharTok{+}
    \FunctionTok{geom\_pointdensity}\NormalTok{(}\AttributeTok{method =} \StringTok{"kde2d"}\NormalTok{) }\SpecialCharTok{+}
    \FunctionTok{scale\_color\_viridis\_c}\NormalTok{() }\SpecialCharTok{+}
    \CommentTok{\# scale\_size\_continuous(range = c(0.001, 4)) + }
    \FunctionTok{scale\_x\_log10}\NormalTok{() }\SpecialCharTok{+}
    \FunctionTok{scale\_y\_continuous}\NormalTok{(}\AttributeTok{breaks =} \FunctionTok{c}\NormalTok{(}\DecValTok{0}\NormalTok{, }\DecValTok{1}\NormalTok{)) }\SpecialCharTok{+} 
    \FunctionTok{xlab}\NormalTok{(}\StringTok{"Mean population size"}\NormalTok{) }\SpecialCharTok{+}
    \FunctionTok{ylab}\NormalTok{(}\StringTok{"Speciation probability"}\NormalTok{) }\SpecialCharTok{+}
    \FunctionTok{theme}\NormalTok{(}\AttributeTok{panel.grid.minor.y =} \FunctionTok{element\_blank}\NormalTok{()) }\SpecialCharTok{+}
    \FunctionTok{geom\_smooth}\NormalTok{(}\AttributeTok{method =}\NormalTok{ glm, }\AttributeTok{formula =}\NormalTok{ y }\SpecialCharTok{\textasciitilde{}}\NormalTok{ x }\SpecialCharTok{+} \FunctionTok{I}\NormalTok{(x}\SpecialCharTok{\^{}}\DecValTok{2}\NormalTok{), }
                \AttributeTok{method.args =} \FunctionTok{list}\NormalTok{(}\AttributeTok{family =} \StringTok{"binomial"}\NormalTok{), }\AttributeTok{color =} \StringTok{"black"}\NormalTok{)}

\NormalTok{sim\_fig}
\end{Highlighting}
\end{Shaded}

\pandocbounded{\includegraphics[keepaspectratio]{ab_supp_files/figure-pdf/make-sim-fig-1.pdf}}

\section*{References}\label{references}
\addcontentsline{toc}{section}{References}

\phantomsection\label{refs}
\begin{CSLReferences}{1}{1}
\bibitem[\citeproctext]{ref-quarto}
Allaire, J. J., Teague, C., Scheidegger, C., Xie, Y., Dervieux, C. and
Woodhull, G. 2025.
\href{https://doi.org/10.5281/zenodo.5960048}{{Quarto}}.

\end{CSLReferences}




\end{document}
